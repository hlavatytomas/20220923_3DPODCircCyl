\section{Conclusion}
\label{sec:concl}
% Note (MI): to be re-thought
Direct comparison of experimental results and results of mathematical modeling of the turbulent wake behind a circular cylinder {in a cross flow} is carried out in the paper. {A special emphasis is made on a definition of criteria to be compared in order to validate not only the flow statistics, but also its dynamics.}

{To validate the flow-field statistics,} the {time-}averaged characteristics were compared, i.e. {the} time-mean velocity {and turbulent kinetic energy} field{s} topolog{ies} {on selected planes of measurements, streamwise PoM1 and transverse PoM2, respectively.}

%~ The attention is paid to definition of criteria to be compared to take into account {the} statistical, averaged{,} topology as well as the flow dynamics. To compare the flow-field statistics, the {time-}averaged characteristics were compared, i.e. {the} time-mean velocity {and turbulent kinetic energy} field{s} topolog{ies} {on selected planes of measurements.}% and second order statistical moments – variances distribution.

% Note (MI): just copy and paste below
{The flow dynamics was first examined via a comparison of turbulence power spectra sampled in two probe points. Next,} the dynamical topology of the velocity field was validated with {the} help of {the} POD analysis technique. The kinetic energy containing modes were evaluated {from PoM1 and PoM2 for} both experimental and CFD data and compared as for theirs topologies {and temporal characteristics}.% For the comparison of time domain characteristics, the spectra of velocity components in selected locations within the wake were compared.

{Despite detected} minor differences in some characteristics evaluated from experiments and CFD, {the} overall agreement is very good for {both the flow} statistics and dynamical characteristics. {Consequently, we were able to leverage the fact that fully three-dimensional data are available from the simulation and to perform a 3D POD analysis of the studied flow.}

{The first two 3D POD modes, which capture $70\,$\% of the overall system turbulence kinetic energy, contain coherent predominantly two-dimensional vortex structures parallel to the cylinder and exhibit laminar-like quasi-periodic dynamics. The next four modes exhibit the same dynamic behavior as the first two and are responsible for rotating the coherent structures to form oblique vortex shedding. The seventh and eight 3D POD modes still show somehow regular dynamic behavior. However, with the applied methods, we were not able to disclose their effects on the flow original flow.}

%~ The first six 3D POD modes, which together capture $80\,$\% of the overall turbulence kinetic energy of the flow, contain coherent predominantly two-dimensional vortex structures. The seventh and higher modes are more affected by the three-dimensional turbulence.
%~ All validations were performed in selected planes, streamwise PoM1 and spanwise PoM2, respectively.
%~ We detected minor differences in some characteristics evaluated from experiments and CFD, overall agreement is very good both for statistics and dynamical characteristics.

% section concl (end)
